% the doc class is a modified ITE paper template
\documentclass{two-col-epfl}

% importing bibliography packages and files
\usepackage[
backend=biber,
style=numeric,
sorting=ynt
]{biblatex}
\addbibresource{Bibliography.bib}
\usepackage[colorlinks=true,linkcolor=blue,citecolor=blue]{hyperref}%
\usepackage{subfiles}
\usepackage{bm}
\usepackage{amssymb} %math symbols
\usepackage{mathtools} %more math stuff
\usepackage{amsthm} %theorems, proofs and lemmas

%% Theorem notation
\newtheorem{theorem}{Theorem}[section]
\newtheorem{corollary}{Corollary}[theorem]
\newtheorem{lemma}[theorem]{Lemma}
\newtheorem{problem}{Problem}[section]
\newtheorem{definition}{Definition}[section]
\newtheorem{property}{Property}[section]
\newtheorem{notation}{Notation}[section]
\newtheorem{observation}{Observation}[section]


\begin{document}

\headertitle{Semester Project Report}
\footertitle{EPFL - Biorobotics laboratory - Semester Project Proposal Draft}

%% generating the title section
\title{Learning Motor Policies with Time Continuous Neural Networks}

\author{Titouan Renard $^1$}

\address{\add{1}{MT-RO, 272257}}
\maketitle

% core text


% ---------- Abstract ----------


% ---------- Introduction ----------



\textbf{Abstract}
\textit{Time-Continuous Neural Networks provide an effective framework for the modeling of dynamical systems \cite{Chen2018NeuralOD} and are natural candidates for continuous-control tasks. They are closely related to the dynamics of non-spiking neurons, which gives further justification to investigate their use in control \cite{Lechner2020NeuralCP}. The following document describes the work and results obtained while investigating the use of such neural networks during a semester project jointly supervised by EPFL's BIOROB and LCN labs.}

\section*{Introduction}

We first consider time-continuous neural networks, and the main ideas of reinforcement learning, then we discuss implementation details, and finally we go go through the results obtained during the project.

\section{Time-Continuous Neural Networks}
\subfile{sections/CTN.tex}

\section{Policy Gradient Methods and Reinforcement Learning}
\subfile{sections/RL.tex}

\section{Implementation of CTNN in a Reinforcement Learning framework}
\subfile{sections/implementation.tex}

\section{Results and Discussion}
\subfile{sections/results.tex}


\printbibliography

\end{document}
